The empirical and the spatial networks for site A are denoted as $E_A$ and $S_A$ respectively. The first thing that we did was to see whether we could find a subgraph-isomorphism between $E_A$ and an $S_A$ produced from our spatial model. 
\subsection{Methods}
To achieve this, we investigated various different algorithms for determining the existence of a subgraph-isomorphism, and also outputting the mapping between the two vertex sets if the mapping exists. 
\\\\* Amongst these algorithms were:
\begin{itemize}
    \item Ullmann's Algorithm \cite{Ullmann1976}
    \item The VF \cite{cordella1999performance}, VF2 \cite{Vento2004}, VF2Plus \cite{Carletti2015}, and VF3 \cite{carletti2017introducing} algorithms
    \item The VF2++ algorithm \cite{Juttner2018}
\end{itemize}
\\\\* We chose to use the VF2 algorithm as it is well-established, and has implementations in Python. 
We plan to accomplish this by:
\begin{enumerate}
    \item Creating contact networks using our spatial model, as in Chapter 3.
    \item Then we measure the "distance" between our spatial networks and the empirical networks.
    \item We measure distance by looking at:
    \begin{itemize}
        \item Monomorphisms (subgraph isomorphisms)
        \item Graph edit distance
        \item Similarity score
        \item RMSE between certain network metrics for empirical and spatial networks
    \end{itemize}
\end{enumerate}

\subsection{Monomorphisms}
We researched many different algorithms to 